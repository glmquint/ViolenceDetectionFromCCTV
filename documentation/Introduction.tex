\chapter{Introduction}
The goal of this project is to develop a CNN capable of recognizing the presence of violence in the everyday CCTV footage in a binary detection system (violence/non-violence). In the world of today it is fundamental for police and other forces to have a quick response in case of violent and criminal behavior especially when the people involved cannot dial the emergency lines due to immediate danger. The main idea it would be for these kind of AI algorithm to process enormous amount of video and signal it in case of foul play in real time. This project presented itself as a very difficult one due to the "\textit{dirtiness}" of the data-set used that, due to the fact the it was composed of real life CCTV footage, was not standardized, contained a lot o noise and many more other problems that will be explained in the next paragraph.

\paragraph{Dataset presentation}
The data-set was obtained by publicly available sources:
\begin{itemize}
	\item Smart-City CCTV Violence Detection Dataset\footnote{\url{https://www.kaggle.com/datasets/toluwaniaremu/smartcity-cctv-violence-detection-dataset-scvd/data}}: \~1Gb
	\item seymanurakti/fight-detection-surv-dataset\footnote{\url{https://github.com/seymanurakti/fight-detection-surv-dataset}}: \~100Mb
\end{itemize}

The first one was extracted from Kaggle and the data-set was divided into three sub-category (Non-Violence/Violence/Weapon-Violence)\footnote{Taken from version 2, now version 3 is available}, the second one was taken from a GitHub repository and the videos where divided into two sub folder (fight/noFight). The whole data-set was then divided into two main categories: violence and non-violence.

As previously said the problem, which is by itself quite difficult, was made harder due to the poor usability of the data-set that did not give any information on the videos aside form the category like: action frame and bounding boxes. Moreover the first set of videos received a usability score of 6.88 which represent a community rating of the data-set as a whole, for example some video contained writing in the screen, others were not CCTV footage, but live news one and some were recording through a phone of a screen with the footage. In addition the data-set was not standardized at all with videos with variable length and some greyscaled, some full color.
